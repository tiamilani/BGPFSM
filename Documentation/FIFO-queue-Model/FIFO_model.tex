\documentclass[10pt,conference,letterpaper]{IEEEtran}

\usepackage[utf8]{inputenc}
\usepackage[T1]{fontenc}
\usepackage{silence}\WarningsOff[latexfont]

\usepackage{amsmath}
\usepackage{amsfonts}
\usepackage{amssymb}
\usepackage{amsthm}

\usepackage{graphicx}
\graphicspath{images/}
\usepackage{cite}
\usepackage{url}
% \usepackage{subfig}
\usepackage{subcaption}
\usepackage{float}
\usepackage[ruled,vlined,linesnumbered]{algorithm2e}
\SetKwProg{Fn}{Event}{}{}
\SetKw{And}{and}
\usepackage[binary-units,per-mode=symbol]{siunitx}
\sisetup{list-final-separator = {, and },detect-weight=true, detect-family=true}
\usepackage{booktabs}
\usepackage{pifont}
\usepackage{microtype}
\usepackage{textcomp}
\usepackage[american]{babel}
\usepackage[capitalise]{cleveref}
\def\figname{\csname cref@figure@name\endcsname\xspace}
\def\tabname{\csname cref@table@name\endcsname\xspace}
\def\secname{\csname cref@section@name\endcsname\xspace}
\def\eqpname{\csname cref@equation@name@plural\endcsname\xspace}
\crefname{algorithm}{Listing}{Lists.}
\Crefname{algorithm}{Listing}{Listings}
\SetAlgorithmName{Listing}{Listing}{List of Listings}
\crefname{lstlisting}{listing}{listings}
\Crefname{lstlisting}{Listing}{Listings}
\usepackage{xspace}
\usepackage{hyphenat}
\usepackage[draft,inline,nomargin,index]{fixme}
\fxsetup{theme=color}
\usepackage{grffile}
\usepackage{xfrac}
\usepackage{multirow}
%\usepackage[para]{footmisc}
\usepackage[font={small}]{caption}

\usepackage{tikz}
\usetikzlibrary{calc,shapes,arrows,fit,positioning}

\usepackage{listings}
\lstset{
   language=sh,
   columns=fixed,
   breaklines=true,
   breakatwhitespace=true,
   prebreak=\textbackslash,
   basicstyle=\ttfamily\small,
   showstringspaces=false,
   upquote=true,
   keywordstyle=\ttfamily\small
}

\usepackage{color}
\definecolor{gray}{rgb}{0.4,0.4,0.4}
\definecolor{darkblue}{rgb}{0.0,0.0,0.6}
\definecolor{cyan}{rgb}{0.0,0.6,0.6}


\lstdefinelanguage{XML}
{
  morestring=[b]",
  morestring=[s]{>}{<},
  morecomment=[s]{<?}{?>},
  stringstyle=\color{black},
  identifierstyle=\color{darkblue},
  keywordstyle=\color{cyan},
  morekeywords={xmlns,version,type}% list your attributes here
}


% fix cleveref and breqn
\makeatletter
\let\cref@old@eq@setnumberOld\eq@setnumber
\def\eq@setnumber{%
\cref@old@eq@setnumberOld%
\cref@constructprefix{equation}{\cref@result}%
\protected@xdef\cref@currentlabel{%
[equation][\arabic{equation}][\cref@result]\p@equation\eq@number}}
\makeatother

% reduce verbatim font size
\usepackage{etoolbox}
\makeatletter
\patchcmd{\@verbatim}
  %%%{\verbatim@font} %% blow up TexMaker formatting ???!!! 
  {\verbatim@font\small}
  {}{}
\makeatother

\RequirePackage{xstring}
\RequirePackage{xparse}
\RequirePackage[index=true]{acro}
\NewDocumentCommand\acrodef{mO{#1}mG{}}{\DeclareAcronym{#1}{short={#2}, long={#3}, #4}}
\NewDocumentCommand\acused{m}{\acuse{#1}}


\acrodef{ADV}{advertisement}
\acrodef{AS}{Autonomous System}{short-plural=es}
\acrodef{BGP}{Border Gateway Protocol}
\acrodef{BIRD}{BGP Internet Routing Daemon}
\acrodef{DPC}{Destination Partial Centrality}
\acrodef{eBGP}{Exterior BGP}
\acrodef{ERP}{Exterior Routing Protocol}
\acrodef{IoF}{Internet on FIRE}
\acrodef{IP}{Internet Protocol}
\acrodef{MRAI}{Minimum Route Advertisement Interval}
\acrodef{NH}{Next Hop}
\acrodef{RFC}{Request For Comment} 
\acrodef{TCP}{Transmission Control Protocol}
\acrodef{FSM}{Finite State Machine}
\acrodef{SR}{SemiRings}

\newcommand\useallac{
\acused{IP}
\acused{TCP}
\acused{RFC}
}

\useallac

\theoremstyle{definition}
\newtheorem{definition}{Definition}

\theoremstyle{remark}
\newtheorem*{remark}{Remark}

\newcommand{\figwidthfour}{0.78}
\newcommand{\figwidth}{0.78}
\newcommand{\figvspace}{-1.5em}
\newcommand{\update}{\texttt{UPDATE}\xspace}
\newcommand{\nodeset}{\ensuremath{\mathcal{V}}\xspace}
\newcommand{\edgeset}{\ensuremath{\mathcal{E}}\xspace}
\newcommand{\graphg}{\ensuremath{\mathcal{G}}\xspace}
\newcommand{\graph}{\ensuremath{\mathcal{\graphg(\nodeset,\edgeset)}}\xspace}
\newcommand{\natpernat}{\ensuremath{\mathcal{\mathbb{N} \times \mathbb{N}}}\xspace}

\newcommand{\semiringset}{\ensuremath{\matchal{S}}\xspace}
\newcommand{\semiringchoice}{\ensuremath{\matchal{\oplus}}\xspace}
\newcommand{\semiringfunctions}{\ensuremath{\matchal{F}}\xspace}
\newcommand{\semiringempty}{\ensuremath{\matchal{\overline{\rm 0}}}\xspace}
\newcommand{\semiringinvalid}{\ensuremath{\matchal{\overline{\rm \infty}}}\xspace}
\newcommand{\semiring}{\ensuremath{\mathcal{(\semiringset, \semiringchoice, \semiringfunctions, \semiringempty, \semiringinvalid)}}\xspace}
\newcommand{\invalidfunction}{\ensuremath{f_\semiringinvalid}\xspace}

\newcommand{\Amatrix}{\ensuremath{\mathcal{A}}\xspace}
\newcommand{\Amatrixelem}{\ensuremath{\Amatrix_{ij}}\xspace}

\newcommand{\invalidpath}{\ensuremath{\mathcal{\bot}}\xspace}
\newcommand{\pathset}{\ensuremath{\mathcal{P}}\xspace}

\newcommand{\pathalgebra}[1]{\ensuremath{\mathcal{\mathbb{P}\mathbb{A}(#1)} = (((\semiringset - \{\infty\})
	\times \pathset) \cup \{\infty^{'}\}, \semiringchoice^{'}, \semiringfunctions^{'}, 
	(\semiringempty, []), \infty^{'})}\xspace}
\newcommand{\pathalgebrashort}[1]{\ensuremath{\mathcal{\mathbb{P}\mathbb{A}(#1)}}\xspace}

\newcommand{\concatenation}{\ensuremath{\mathcal{\hat{::}}}\xspace}

\newcommand{\semiringpath}{\ensuremath{\mathcal{(\semiringset, \semiringchoice, \semiringfunctions, \semiringempty, \semiringinvalid, \text{\textit{path}})}}\xspace}

\newcommand{\bufferset}{\ensuremath{\mathcal{B}}\xpsace}

\newcommand{\bufferalgebra}[1]{\ensuremath{\mathcal{\mathbb{B}\mathbb{A}(#1)} = (((\semiringset - \{\infty\})
	\times \bufferset) \cup \{\infty^{'}\}, \semiringchoice^{'}, \semiringfunctions^{'}, 
	(\semiringempty, []), \infty^{'})}\xspace}
\newcommand{\bufferalgebrashort}[1]{\ensuremath{\mathcal{\mathbb{B}\mathbb{A}(#1)}}\xspace}
\newcommand{\semiringbuffer}{\ensuremath{\mathcal{(\semiringset, \semiringchoice, \semiringfunctions, \semiringempty, \semiringinvalid, \text{\textit{buffer}})}}\xspace}

\newcommand{\setofmatrixes}{\ensuremath{\mathcal{\mathbb{M}}_n(\semiringset)}\xspace}

\newcommand{\sigmaop}[1]{\ensuremath{\mathcal{\sigma}(#1)_{ij}}\xspace}

\newcommand{\highlevelmodel}{\ensuremath{\Gamma_0}\xspace}

\newcommand{\resultmatrix}{\ensuremath{\mathcal{X}}\xspace}
\newcommand{\currentstatematrix}{\ensuremath{\mathcal{Y}}\xspace}
\newcommand{\Imatrix}{\ensuremath{\mathcal{I}}\xspace}

\IEEEoverridecommandlockouts

\begin{document}

\title{(draft) FIFO queues Model}
\author{
	\IEEEauthorblockN{Mattia Milani\IEEEauthorrefmark{1}}
    \IEEEauthorblockA{\IEEEauthorrefmark{1}Dept. of Information Engineering and Computer Science, University of Trento, Italy}
    \texttt{mattia.milani@studenti.unitn.it}
}


\maketitle

\section{Main Idea}
\label{sec:mainIdea}

Take an already proven model that defines hard-state protocols and implement
on it FIFO queues on the edges.

This document uses models and demonstration from \cite{AgdaHardStateVectoringRouting}.

This document assumes the knowledge of semirings as \semiring to model routing
problems.

\section{Goals}
\label{sec:goals}

The goal of this document is to amplify what demonstrated in \cite{AgdaHardStateVectoringRouting}
introducing a FIFO queue on the edges.

The second goal of this work is to introduce an asynchoronus formalization of this
fourth model.

\section{Network}
\label{sec:network}

A network is represented by a directed graph \graph where \nodeset is a set of
$n$ nodes $\nodeset=\{0,1, ... , n-1\}$ and \edgeset is a set of arcs.
A configuration of \graphg with respect to a routing algebra \semiring is a mapping
from \edgeset to \semiringfunctions.

Such mappings will be represented by an $n \times n$ adjacency matrix \Amatrix where
$\Amatrixelem \in \semiringfunctions$. 

I assume the constant function $\invalidfunction \in \semiringfunctions$ exists that always returns the invalid
weight, function used to represent missing edges.

For example we can have the following graph:

\begin{center}
\begin{tikzpicture}[scale=0.2]
\tikzstyle{every node}+=[inner sep=0pt]
\draw [black] (18.3,-25.7) circle (3);
\draw (18.3,-25.7) node {$0$};
\draw [black] (35.3,-25.7) circle (3);
\draw (35.3,-25.7) node {$2$};
\draw [black] (52,-25.7) circle (3);
\draw (52,-25.7) node {$4$};
\draw [black] (35.3,-40.5) circle (3);
\draw (35.3,-40.5) node {$3$};
\draw [black] (35.3,-10.8) circle (3);
\draw (35.3,-10.8) node {$1$};
\draw [black] (20.56,-23.72) -- (33.04,-12.78);
\fill [black] (33.04,-12.78) -- (32.11,-12.93) -- (32.77,-13.68);
\draw (25.28,-17.76) node [above] {$f_{1}$};
\draw [black] (35.3,-13.8) -- (35.3,-22.7);
\fill [black] (35.3,-22.7) -- (35.8,-21.9) -- (34.8,-21.9);
	\draw (34.8,-18.25) node [left] {$f_{3}$};
\draw [black] (21.3,-25.7) -- (32.3,-25.7);
\fill [black] (32.3,-25.7) -- (31.5,-25.2) -- (31.5,-26.2);
	\draw (26.8,-26.2) node [below] {$f_{2}$};
\draw [black] (38.3,-25.7) -- (49,-25.7);
\fill [black] (49,-25.7) -- (48.2,-25.2) -- (48.2,-26.2);
	\draw (43.65,-25.2) node [above] {$f_{4}$};
\draw [black] (49.75,-27.69) -- (37.55,-38.51);
\fill [black] (37.55,-38.51) -- (38.48,-38.35) -- (37.81,-37.61);
	\draw (45.17,-33.59) node [below] {$f_{6}$};
\draw [black] (35.3,-37.5) -- (35.3,-28.7);
\fill [black] (35.3,-28.7) -- (34.8,-29.5) -- (35.8,-29.5);
	\draw (35.8,-33.1) node [right] {$f_{5}$};
\end{tikzpicture}
\end{center}

That has as \Amatrix the one in \cref{matrix:simple_graph_edgematrix}

\begin{equation}
	  \Amatrix=
	  \left[ {\begin{array}{ccccc}
		   \invalidfunction & f_{1} & f_{2} & \invalidfunction & \invalidfunction\\
		   \invalidfunction & \invalidfunction & f_{3} & \invalidfunction & \invalidfunction\\
		   \invalidfunction & \invalidfunction & \invalidfunction & \invalidfunction & f_{4}\\
		   \invalidfunction & \invalidfunction & f_{5} & \invalidfunction & \invalidfunction\\
		   \invalidfunction & \invalidfunction & \invalidfunction & f_{6} & \invalidfunction\\
	  \end{array} } \right]
	  \label{matrix:simple_graph_edgematrix}
\end{equation}

\section{Background}
\label{sec:background}

Brief recap of the models in \cite{AgdaHardStateVectoringRouting}.

In that document are formalized three new models that amplify what represent the
basic high level model that is compleatly abstracted from network concepts.
This model is represented by \highlevelmodel.
In this model the solution to the routing path problem is given by a matrix \resultmatrix
where each element $\resultmatrix_{ij}$ represent the best path from $i$ to $j$.
The solution is computed by iteratively appling the adjacent matrix \Amatrix to
the actual routing state \currentstatematrix (Every router synchrounously chose
the best path extension from it's neighbourhood in state \currentstatematrix).

\begin{equation}
	\highlevelmodel ( \currentstatematrix ) = \Amatrix ( \currentstatematrix) \semiringchoice \Imatrix 
	\label{eq:highlevelround}
\end{equation}

\begin{equation}
	\resultmatrix = \Amatrix ( \resultmatrix ) \semiringchoice \Imatrix 
	\label{eq:highlevelsolution}
\end{equation}

\Cref{eq:highlevelsolution} represent the solution to the routing problem using
the high level model, one single round of the model is defined by \Cref{eq:highlevelround}.

The 

\bibliographystyle{IEEEtran}
\bibliography{references}

\end{document}
