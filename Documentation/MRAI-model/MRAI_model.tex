\documentclass[10pt,conference,letterpaper]{IEEEtran}

\usepackage[utf8]{inputenc}
\usepackage[T1]{fontenc}
\usepackage{silence}\WarningsOff[latexfont]

\usepackage{amsmath}
\usepackage{amsfonts}
\usepackage{amssymb}
\usepackage{amsthm}

\usepackage{graphicx}
\graphicspath{images/}
\usepackage{cite}
\usepackage{url}
% \usepackage{subfig}
\usepackage{subcaption}
\usepackage{float}
\usepackage[ruled,vlined,linesnumbered]{algorithm2e}
\SetKwProg{Fn}{Event}{}{}
\SetKw{And}{and}
\usepackage[binary-units,per-mode=symbol]{siunitx}
\sisetup{list-final-separator = {, and },detect-weight=true, detect-family=true}
\usepackage{booktabs}
\usepackage{pifont}
\usepackage{microtype}
\usepackage{textcomp}
\usepackage[american]{babel}
\usepackage[capitalise]{cleveref}
\def\figname{\csname cref@figure@name\endcsname\xspace}
\def\tabname{\csname cref@table@name\endcsname\xspace}
\def\secname{\csname cref@section@name\endcsname\xspace}
\def\eqpname{\csname cref@equation@name@plural\endcsname\xspace}
\crefname{algorithm}{Listing}{Lists.}
\Crefname{algorithm}{Listing}{Listings}
\SetAlgorithmName{Listing}{Listing}{List of Listings}
\crefname{lstlisting}{listing}{listings}
\Crefname{lstlisting}{Listing}{Listings}
\usepackage{xspace}
\usepackage{hyphenat}
\usepackage[draft,inline,nomargin,index]{fixme}
\fxsetup{theme=color}
\usepackage{grffile}
\usepackage{xfrac}
\usepackage{multirow}
%\usepackage[para]{footmisc}
\usepackage[font={small}]{caption}

\usepackage{tikz}
\usetikzlibrary{calc,shapes,arrows,fit,positioning}

\usepackage{listings}
\lstset{
   language=sh,
   columns=fixed,
   breaklines=true,
   breakatwhitespace=true,
   prebreak=\textbackslash,
   basicstyle=\ttfamily\small,
   showstringspaces=false,
   upquote=true,
   keywordstyle=\ttfamily\small
}

\usepackage{color}
\definecolor{gray}{rgb}{0.4,0.4,0.4}
\definecolor{darkblue}{rgb}{0.0,0.0,0.6}
\definecolor{cyan}{rgb}{0.0,0.6,0.6}


\lstdefinelanguage{XML}
{
  morestring=[b]",
  morestring=[s]{>}{<},
  morecomment=[s]{<?}{?>},
  stringstyle=\color{black},
  identifierstyle=\color{darkblue},
  keywordstyle=\color{cyan},
  morekeywords={xmlns,version,type}% list your attributes here
}


% fix cleveref and breqn
\makeatletter
\let\cref@old@eq@setnumberOld\eq@setnumber
\def\eq@setnumber{%
\cref@old@eq@setnumberOld%
\cref@constructprefix{equation}{\cref@result}%
\protected@xdef\cref@currentlabel{%
[equation][\arabic{equation}][\cref@result]\p@equation\eq@number}}
\makeatother

% reduce verbatim font size
\usepackage{etoolbox}
\makeatletter
\patchcmd{\@verbatim}
  %%%{\verbatim@font} %% blow up TexMaker formatting ???!!! 
  {\verbatim@font\small}
  {}{}
\makeatother

\RequirePackage{xstring}
\RequirePackage{xparse}
\RequirePackage[index=true]{acro}
\NewDocumentCommand\acrodef{mO{#1}mG{}}{\DeclareAcronym{#1}{short={#2}, long={#3}, #4}}
\NewDocumentCommand\acused{m}{\acuse{#1}}


\acrodef{ADV}{advertisement}
\acrodef{AS}{Autonomous System}{short-plural=es}
\acrodef{BGP}{Border Gateway Protocol}
\acrodef{BIRD}{BGP Internet Routing Daemon}
\acrodef{DPC}{Destination Partial Centrality}
\acrodef{eBGP}{Exterior BGP}
\acrodef{ERP}{Exterior Routing Protocol}
\acrodef{IoF}{Internet on FIRE}
\acrodef{IP}{Internet Protocol}
\acrodef{MRAI}{Minimum Route Advertisement Interval}
\acrodef{NH}{Next Hop}
\acrodef{RFC}{Request For Comment} 
\acrodef{TCP}{Transmission Control Protocol}
\acrodef{FSM}{Finite State Machine}
\acrodef{SR}{SemiRings}

\newcommand\useallac{
\acused{IP}
\acused{TCP}
\acused{RFC}
}

\useallac

\theoremstyle{definition}
\newtheorem{definition}{Definition}

\theoremstyle{remark}
\newtheorem*{remark}{Remark}

\newcommand{\figwidthfour}{0.78}
\newcommand{\figwidth}{0.78}
\newcommand{\figvspace}{-1.5em}
\newcommand{\update}{\texttt{UPDATE}\xspace}
\newcommand{\nodeset}{\ensuremath{\mathcal{V}}\xspace}
\newcommand{\edgeset}{\ensuremath{\mathcal{E}}\xspace}
\newcommand{\graphg}{\ensuremath{\mathcal{G}}\xspace}
\newcommand{\graph}{\ensuremath{\mathcal{\graphg(\nodeset,\edgeset)}}\xspace}
\newcommand{\natpernat}{\ensuremath{\mathcal{\mathbb{N} \times \mathbb{N}}}\xspace}

\newcommand{\semiringset}{\ensuremath{\matchal{S}}\xspace}
\newcommand{\semiringchoice}{\ensuremath{\matchal{\oplus}}\xspace}
\newcommand{\semiringfunctions}{\ensuremath{\matchal{F}}\xspace}
\newcommand{\semiringempty}{\ensuremath{\matchal{\overline{\rm 0}}}\xspace}
\newcommand{\semiringinvalid}{\ensuremath{\matchal{\overline{\rm \infty}}}\xspace}
\newcommand{\semiring}{\ensuremath{\mathcal{(\semiringset, \semiringchoice, \semiringfunctions, \semiringempty, \semiringinvalid)}}\xspace}
\newcommand{\invalidfunction}{\ensuremath{f_\semiringinvalid}\xspace}

\newcommand{\Amatrix}{\ensuremath{\mathcal{X}}\xspace}
\newcommand{\Amatrixelem}{\ensuremath{\Amatrix_{ij}}\xspace}

\newcommand{\invalidpath}{\ensuremath{\mathcal{\bot}}\xspace}
\newcommand{\pathset}{\ensuremath{\mathcal{P}}\xspace}

\newcommand{\pathalgebra}[1]{\ensuremath{\mathcal{\mathbb{P}\mathbb{A}(#1)} = (((\semiringset - \{\infty\})
	\times \pathset) \cup \{\infty^{'}\}, \semiringchoice^{'}, \semiringfunctions^{'}, 
	(\semiringempty, []), \infty^{'})}\xspace}
\newcommand{\pathalgebrashort}[1]{\ensuremath{\mathcal{\mathbb{P}\mathbb{A}(#1)}}\xspace}

\newcommand{\concatenation}{\ensuremath{\mathcal{\hat{::}}}\xspace}

\newcommand{\semiringpath}{\ensuremath{\mathcal{(\semiringset, \semiringchoice, \semiringfunctions, \semiringempty, \semiringinvalid, \text{\textit{path}})}}\xspace}

\newcommand{\bufferset}{\ensuremath{\mathcal{B}}\xpsace}

\newcommand{\bufferalgebra}[1]{\ensuremath{\mathcal{\mathbb{B}\mathbb{A}(#1)} = (((\semiringset - \{\infty\})
	\times \bufferset) \cup \{\infty^{'}\}, \semiringchoice^{'}, \semiringfunctions^{'}, 
	(\semiringempty, []), \infty^{'})}\xspace}
\newcommand{\bufferalgebrashort}[1]{\ensuremath{\mathcal{\mathbb{B}\mathbb{A}(#1)}}\xspace}
\newcommand{\semiringbuffer}{\ensuremath{\mathcal{(\semiringset, \semiringchoice, \semiringfunctions, \semiringempty, \semiringinvalid, \text{\textit{buffer}})}}\xspace}

\newcommand{\setofmatrixes}{\ensuremath{\mathcal{\mathbb{M}}_n(\semiringset)}\xspace}

\newcommand{\sigmaop}[1]{\ensuremath{\mathcal{\sigma}(#1)_{ij}}\xspace}

\IEEEoverridecommandlockouts

\begin{document}

\title{(draft) MRAI Model}
\author{
	\IEEEauthorblockN{Mattia Milani\IEEEauthorrefmark{1}}
    \IEEEauthorblockA{\IEEEauthorrefmark{1}Dept. of Information Engineering and Computer Science, University of Trento, Italy}
    \texttt{mattia.milani@studenti.unitn.it}
}


\maketitle

\section{Main Idea}
\label{sec:mainIdea}

This document has to bee take as my flow of consiousness, it will be
probably written again, corrected and changed many times.

The main idea behind the \ac{MRAI} model that I have in mind is that: 
In \cite{daggitt2018rate} has been shown how to model and algebricaly solve
problems with \ac{SR} like hop-count shortest-path and even policies.
One of the main ideas is to consider the network as a matrix $A$ where each
row represent a node $i$ and each column represent a function to reach node
$j$.
The adjacency matrix $A$ permits to have a mastrix $X$ that represent the state of
the network, this matrix $X$ evolves in time, at each step a routing algebra is applied to each
edge solving the local shortest path problem.
At some point, a further step would not make the matrix change, so convergence
has been reached.

My is an asynchronous approach to the matrix evolution.
My Idea is to don't apply the algebra to the whole matrix $A$ but apply it
locally to single edges selecting the best route in the knowledge buffer
of the node neighbourhood.
The algebra would select the best possible path in the vector of possibilities
in the nodes and then introudce this coice in the set of possibilities of 
the edge.
The step can be executed indipendently from each other edge in the network.
At the end, when no steps produce changes the network is converged.

\section{Goals}
\label{sec:goals}

The main goal is to be able to model \ac{MRAI} and solve problems caused by it.

\section{Semirings, Routing algebra}
\label{sec:routing_algebra}

I should start in order, first of all I have to define what is a routing algebra.

\theoremstyle{definition}
\begin{definition}
	A routing algebra is a tuple \semiring where:
	\begin{itemize}
			\item \semiringset the set of weights with $\semiringempty \in \semiringset$
				and $\semiringinvalid \in \semiringset$,
			\item \semiringchoice: $\semiringset X \semiringset \to \semiringset$
				is the choice operator
			\item \semiringfunctions is a set of functions from $\semiringset \to \semiringset$
	\end{itemize}
	such that:
	\semiringchoice is Associative, commutative and selective, \semiringempty is
	an annhilator for \semiringchoice, \semiringinvalid is an identitity function for
	\semiringchoice and a fixed point $\forall f \in \semiringfunctions$
\end{definition}

It's possible to define a total order between elements.

\begin{equation}
	x \leq y \triangleq x \semiringchoice y = x \quad
	x < y \triangleq x \leq y \land x \neq y

	x \leq y \iff x \semiringchoice y = x
\end{equation}

each function \semiringfunctions represents a procedure to transform a route's
weight when that route is extended along and edge.
Elements of \semiringfunctions will be used to label edges, so we have
arc functions.

The algebra is distributive if:

\begin{equation}
	\forall f \in \semiringfunctions, x,y \in \semiringset: f(x \semiringchoice y) = f(x) \semiringchoice f(y) 
\end{equation}

The algebra is increasing if:

\begin{equation}
	\forall f \in \semiringfunctions, x \in \semiringset: x \leq f(x)
\end{equation}

The algebra is strictly increasing if:

\begin{equation}
	\forall f \in \semiringfunctions, x \in \semiringset: x \neq \semiringinvalid \Rightarrow  x < f(x)
\end{equation}

\section{Network}
\label{sec:network}

A network is represented by a directed graph \graph where \nodeset is a set of
$n$ nodes $\nodeset=\{0,1, ... , n-1\}$ and \edgeset is a set of arcs.
A configuration of \graphg with respect to a routing algebra \semiring is a mapping
from \edgeset to \semiringfunctions.

Such mappings will be represented by an $n \times n$ adjacency matrix \Amatrix where
$\Amatrixelem \in \semiringfunctions$. 

I assume the constant function $\invalidfunction \in \semiringfunctions$ exists that always returns the invalid
weight, function used to represent missing edges.

For example we can have the following graph:

\begin{center}
\begin{tikzpicture}[scale=0.2]
\tikzstyle{every node}+=[inner sep=0pt]
\draw [black] (18.3,-25.7) circle (3);
\draw (18.3,-25.7) node {$0$};
\draw [black] (35.3,-25.7) circle (3);
\draw (35.3,-25.7) node {$2$};
\draw [black] (52,-25.7) circle (3);
\draw (52,-25.7) node {$4$};
\draw [black] (35.3,-40.5) circle (3);
\draw (35.3,-40.5) node {$3$};
\draw [black] (35.3,-10.8) circle (3);
\draw (35.3,-10.8) node {$1$};
\draw [black] (20.56,-23.72) -- (33.04,-12.78);
\fill [black] (33.04,-12.78) -- (32.11,-12.93) -- (32.77,-13.68);
\draw (25.28,-17.76) node [above] {$f_{1}$};
\draw [black] (35.3,-13.8) -- (35.3,-22.7);
\fill [black] (35.3,-22.7) -- (35.8,-21.9) -- (34.8,-21.9);
	\draw (34.8,-18.25) node [left] {$f_{3}$};
\draw [black] (21.3,-25.7) -- (32.3,-25.7);
\fill [black] (32.3,-25.7) -- (31.5,-25.2) -- (31.5,-26.2);
	\draw (26.8,-26.2) node [below] {$f_{2}$};
\draw [black] (38.3,-25.7) -- (49,-25.7);
\fill [black] (49,-25.7) -- (48.2,-25.2) -- (48.2,-26.2);
	\draw (43.65,-25.2) node [above] {$f_{4}$};
\draw [black] (49.75,-27.69) -- (37.55,-38.51);
\fill [black] (37.55,-38.51) -- (38.48,-38.35) -- (37.81,-37.61);
	\draw (45.17,-33.59) node [below] {$f_{6}$};
\draw [black] (35.3,-37.5) -- (35.3,-28.7);
\fill [black] (35.3,-28.7) -- (34.8,-29.5) -- (35.8,-29.5);
	\draw (35.8,-33.1) node [right] {$f_{5}$};
\end{tikzpicture}
\end{center}

That has as \Amatrix the one in \cref{matrix:simple_graph_edgematrix}

\begin{equation}
	  \Amatrix=
	  \left[ {\begin{array}{ccccc}
		   \invalidfunction & f_{1} & f_{2} & \invalidfunction & \invalidfunction\\
		   \invalidfunction & \invalidfunction & f_{3} & \invalidfunction & \invalidfunction\\
		   \invalidfunction & \invalidfunction & \invalidfunction & \invalidfunction & f_{4}\\
		   \invalidfunction & \invalidfunction & f_{5} & \invalidfunction & \invalidfunction\\
		   \invalidfunction & \invalidfunction & \invalidfunction & f_{6} & \invalidfunction\\
	  \end{array} } \right]
	  \label{matrix:simple_graph_edgematrix}
\end{equation}

The definition of path must be independent from the definition of the current
graph, we can have paths that are not corresponding to the current graph \graphg.

A \textit{path} $p = [(v_{1}, v^{'}_{1}), \xspace ... \xspace , (v_{m}, v^{'}_{m})]$
is a (possibily empty) sequence of arcs such that $v^{'}_{i} = v_{i+1}$ for all 
$0 \leq i < m$. Arcs are not members of \edgeset but are arbitrary members of
\natpernat.

I will assume the same notational convenience on paths as \cite{daggitt2018rate}.
$p = []$ represent the empty path, $(v_1, v^{'}_1) :: q$ when $0 \leq i < m$
where $q = [(v_{2}, v^{'}_{2}), \xspace ... \xspace , (v_{k}, v^{'}_{k})]$,
a path is simple if it never visit a vertex mre than once. the \invalidpath simbol
to represent the invalid path, \pathset represent the set of paths, $|p|$ is the
length of the path $p$

\section{Path algebras}
\label{sec:pathalgebra}

We are interested in protocols which track the path tat routes are generated along and
remove any looping path. Given the routing algebra $\mathcal{A} = \semiring$ we can
add path to the algebra as follows:

\pathalgebra{A}

In \pathalgebrashort{A} weights are in form $(s, p)$ or $\infty^{'}$.
where $s \in \semiringset - \{\infty\}$ and $p \in \pathset$.
$\forall (s, p) : (s, p) \semiringchoice^{'} \infty^{'} = (s, p)$, non-$\infty^{'}$
rutes are compared lexicographically:

\begin{equation}
  (s_1, p_1) \semiringchoice (s_2, p_2) &\triangleq
  \begin{cases}
	  (s_1, p_1)
		  & \text{if $s_1 = (s1 \semiringchoice s2 ) \neq s_2$}, \\
	  (s_2, p_2)
		  & \text{if $s_1 \neq (s1 \semiringchoice s2 ) = s_2$}, \\
	  (s_1, p_1 \hat{\semiringchoice} p_2)
		  & \text{if $s_1 = s_2$},
  \end{cases} 
\end{equation}

where 

\begin{equation}
	p_1 \hat{\semiringchoice} p_2 &\triangleq
  \begin{cases}
	  p_1
		  & \text{if $|p_1| < |p_2|$}, \\
	  p_2
		  & \text{if $|p_1| > |p_2|$}, \\
	  \text{dict$(p_1, p_2)$} 
		  & \text{otherwise},
  \end{cases} 
\end{equation}

dict returns the smallest path in dictionary order.

I define the concatenation operator as in \cite{daggitt2018rate}.
\concatenation takes an arc and a path, $(i,j) \concatenation p$, the result is \invalidpath if
the first node of $p$ is not $j$.

The $\semiringfunctions^{'}$ fr the path-algebra is the collections of policies
functions $g_{f,u,v}$ defined as:

$$g_{f,u,v}(\infty^{'}) = \infty^{'}$$

\begin{equation}
	g_{f,u,v}(s,p) &\triangleq
  \begin{cases}
	  \infty^{'}
		  & \text{if $f(s) = \infty$}, \\
	  \infty^{'}
		  & \text{if $(u, v) \concatenation p = \invalidpath$}, \\
		  (f(s), (u,v) \concatenation p)
		  & \text{otherwise},
  \end{cases} 
\end{equation}

\theoremstyle{definition}
\begin{definition}
	A path algebra is a tuple \semiringpath where:
	\begin{itemize}
			\item \semiring is a routing algebra;
			\item \textit{path}: $\semiringset \rightarrow \pathset$ is a 
				function that returns the path the weight was generated along
	\end{itemize}
	such that:
	\begin{itemize}
		\item[\textit{P1)}] $\text{path}(x) = \invalidpath \iff x = \semiringinvalid$
		\item[\textit{P2)}] $\text{path}(x) = [] \Leftarrow x = \semiringempty$
		\item[\textit{P3)}] $\text{path}(\Amatrixelem(x)) = (i,j) :: \text{\textit{path}}(x) \iff i \notin \text{\textit{path}}(x) \land j = \text{\textit{src}}(\text{\textit{path}}(x))$
	\end{itemize}
\end{definition}

P3) implies that if a path algebra is increasing it is automatically strictly 
increasing.
\pathalgebrashort{A} is a path algebra for any routing algebra,
the path function is defined as $\text{\textit{path}}(\infty^{'}) = \invalidpath$
and $\text{\textit{path}}((s,p)) = p$

\section{Buffer algebra}
\label{sec:buffer}

Thinking about \ac{MRAI} could be easy to think about operation on buffers of paths
rather than the comparison of only paths.

First of all I have to define what is a buffer.
A \textit{buffer} $b = [p_1, p_2, \xspace ... \xpsace, p_k]$ is a (possibily empty)
sequence of paths such that $p_x$ is unique in $b$ for all $ 0 \leq x \leq k$.
For notational convenience I will write $b$ as $[$ $]$ when $k = 0$.
Buffers, for our pourpose, could be only simple.
this restriction could be removed in future.
I will refer to the set of buffers with $\mathcal{B}$ and the length of a buffer $b$
as $|b|$.
I will refer to the sub-buffer $b\{x,y\}$ as the subset $[p_x, \xspace ... \xspace, p_y]$
of $b$ with $0 \leq x < y < |b|$ an invalid subbuffer will return the invalid operator \invalidpath.

For this reason I thought about the following algebra.
Given a routing algebra $\mathcal{A} = \semiring$ one way to add buffers to 
$\mathcal{A}$ is as follows:

\bufferalgebra{A}

In the augmented algebra, \bufferalgebrashort{A}, weights are of the form $\infty^{'}$
or $(s, b)$ where $s \in \semiringset - \{\infty\}$ and $b$ is a buffer.
For all $(s, b)$ we have $(s, b) \semiringchoice^{'}\infty^{'} = \infty^{'}\semiringchoice^{'}(s, b) = (s, p)$.
The result of a comparison is the the resulting best path in the buffer, with $p \in \pathset$.
Non-$\infty^{'}$ routes are compared lexicographically:

\begin{equation}
  (s_1, b_1) \semiringchoice (s_2, b_2) &\triangleq
  \begin{cases}
	  (s_1,\hat{\semiringchoice} b_1)
		  & \text{if $s_1 = (s1 \semiringchoice s2 ) \neq s_2$}, \\
	  (s_2, \hat{\semiringchoice} b_2)
		  & \text{if $s_1 \neq (s1 \semiringchoice s2 ) = s_2$}, \\
	  (s_1, \hat{\semiringchoice}b_1 \hat{\hat{\semiringchoice}} \hat{\semiringchoice}b_2)
		  & \text{if $s_1 = s_2$},
  \end{cases} 
\end{equation}

where 

\begin{equation}
  \hat{\semiringchoice} b &\triangleq
  \begin{cases}
	  \infty^{'}
		  & \text{if $b = []$}, \\
	  \text{\textit{best}}(b)
		  & otherwise
  \end{cases} 
\end{equation}

where, given $p_0$ as the first element of $b = [p_0, p_1, \xspace ... \xspace, p_{|b-1|}]$

\begin{equation}
	\text{\textit{best}}(b) = p_0 \hat{\hat{\semiringchoice}} \text{\textit{best}}(b\{1,|b-1|\})
\end{equation}

This function is used to recursively chose the best path in the buffer.

whith:

\begin{equation}
	p_1 \hat{\hat{\semiringchoice}} p_2 &\triangleq
  \begin{cases}
	  p_1
		  & \text{if $|p_1| < |p_2|$}, \\
	  p_2
		  & \text{if $|p_1| > |p_2|$}, \\
	  \text{dict$(p_1, p_2)$} 
		  & \text{otherwise},
  \end{cases} 
\end{equation}

where dict$(p_1, p_2)$ returns the smallest path in dictionary order.

I define the concatention operator \concatenation that takes a path and a buffer.
The result of $p \concatenation b$ is $b$ if $p$ is already in $b$, otherwaise
it returns $b :: p$. For example:

$$ [(3,4), (4,5)] \concatenation [[(3,4), (4,5)]] = [[(3,4),(4,5)]]$$
$$ [(1,9)] \concatenation [[(3,4), (4,5)]] = [[(3,4), (4,5)],[(1,9)]] $$

If in the buffer is already present a path to the same destination with the same first
edge, the new path should substitute the old path in the buffer.
In this way we have "Implicit Withdrawing".
If a neighbour change idea on how to reach a destination the node can't have
in it's knowledge both paths, but just the last one.

In a more formal way:
\begin{itemize}
	\item Given $(i,j)$ as the edge with the neighbour chosen.
	\item Given $p$ as the best path to reach a destination given by the neighbour $j$.
	\item Given $b$ as the local buffer.
	\item The first concatenation is applied to the path $(i,j) \concatenation p$, if the
concatenation is correct we obtain a path $p^{'} = [(i,j), (j, j_1), \xspace ... \xspace, (j_k, k)]$ 
The destination of this path is the node $k$.
\end{itemize}
To sum up, the resulting buffer of the concatenation will be:

\begin{equation}
	b \concatenation p^{'} &=
  \begin{cases}
	  b
		  & \text{if $p^{'} \in b$}, \\
	  \{b - p_{p}\} \cup p^{'}
		  & \text{if $\begin{split} \exists p_{p} \in b &: p_{p}[0] = p^{'}[0] \\
														& \land dst(p_{p}) = dst(p^{'}) \end{split}$} \\
	  b \cup p^{'}\
		  & \text{otherwise},
  \end{cases} 
\end{equation}


The $\semiringfunctions^{'}$ for the buffer algebra is the collection of policy
functions $g_{f,p}$ which are defined as:

$$g_{f,p}(\infty^{'}) = \infty^{'}$$

\begin{equation}
	g_{f,p}(s,b) &\triangleq
  \begin{cases}
	  \infty^{'}
		  & \text{if $f(s) = \infty$}, \\
	  (f(s), p \concatenation b)
		  & \text{otherwise},
  \end{cases} 
\end{equation}

\theoremstyle{definition}
\begin{definition}
	A buffer algebra is a tuple \semiringbuffer where:
	\begin{itemize}
			\item \semiring is a routing algebra;
			\item \textit{buffer}: $\semiringset \rightarrow \bufferset$ is a 
				function that returns the buffer the weight was generated along
	\end{itemize}
	such that:
	\begin{itemize}
		\item[\textit{B1)}] $\text{buffer}(x) = \invalidpath \iff x = \semiringinvalid$
		\item[\textit{B2)}] $\text{buffer}(x) = [] \Leftarrow x = \semiringempty$
		\item[\textit{B3)}] \begin{equation}
								 \text{\textit{buffer}}(\Amatrixelem(x)) &=
								 \begin{cases}
									 \infty^{'}
									  & \text{if $(i,j) \concatenation path(x) = \invalidpath$}, \\
									 b
									  & \text{if $(i,j) \concatenation path(x) \in b$}, \\
									\begin{split}
										(i,j) & \concatenation path(x) \\
											   & \concatenation b
									\end{split}
 								   	  & \text{otherwise},
 								 \end{cases} 
							\end{equation}
	\end{itemize}
\end{definition}

\bufferalgebrashort{A} is a path algebra for any $\mathcal{A}$. Its buffer function
is defined as $\text{buffer}(\infty^{'}) = \invalidpath$ and $\text{buffer}((s, p)) = p \concatenation b$.

\section{Solving problems}
\label{sec:solving_problems}

Having \graph as a directed graph.
Let $\mathcal{A}$ be the adjacency matrix of \graphg where $\mathcal{A}_{ij} \in \semiringfunctions$
is the weight of the edge from $i$ to $j$.
Missing edges are represented by \invalidfunction.

Let \setofmatrixes be the set of $n \times n$ matrices over \semiringset. Each matrix
$X \in \setofmatrixes$ represent one possible global state of the routing protocol.
The row $X_i$ represent all the current buffers of node $i$ and $X_{ij}$
is node $i$'s current input buffer for reaching node $j$.

Every node has input buffers from which it will retrives information, an input 
buffer for a node is the actual knowledge of all its input neighbours.

Given Equation \num{4} of \cite{daggitt2018rate}, I report an equation that 
will be useful later. 
define the application of $A$ to $X$ as:

$$ A(X)_{ij} \triangleq \semiringchoice_{0 \leq k < n} A_{ik}(X_{kj}) $$

If we apply also the idendity value in $I_{ij}$ we then obtain that:
node i’s new route to j is the best choice out of the extensions
of the best routes offered to it by each knowledge buffer of its neighbours k

\section{Iterative solution}
\label{sec:iterative solution}

I therefore define the function \sigmaop{X} as:

$$ \sigmaop{X} \triangleq A(X)_{ij} \semiringchoice I_{ij} $$

Given $I$ as the identity matrix.

The result of \sigmaop{X} is that $i$ will now extend the buffer in $X_{ij}$ 
with the best path (extended with $i$) contained in all the output buffers of
its neighbourhood.

Refers to \cite{daggitt2018asynchronous} for the asynchronous convergence.
I will explain it to be sure of my vision of it.

\section{Asynchronous convergence}

Now that we have all the buffers that we need we can exploit an asynchronous
convergence system.
I use the model from \"Uresing \& Dubois [insert citation] which assumes a discrete
and linear notion of time $\mathbb{T}$ denoting the times of events of interest
in the network

\theoremstyle{definition}
\begin{definition}
	A Schedule consists of a pair of functions:
	\begin{itemize}
		\item $\alpha : \mathbb{T} \rightarrow 2^{V}$ is the activation function,
			where $\alpha(t)$ is the set of nodes which update their routing table
			at time $t$
		\item $\beta : \mathbb{T} \times V \times V \rightarrow \mathbb{T}$ is the
			data flow function where $\beta(t,i,j)$ is the time at which the 
			information used by node $i$ at time $t$ was sent by node $j$.
	\end{itemize}
	where $V$ is the set of nodes in the network, such that:
	\begin{itemize}
		\item[\textit{S1)}] Every node continues to activate indefinetely
			$$\forall it . \exists k. i \in \alpha(t + k) $$
		\item[\textit{S2)}] information only travels forward in time
			$$\forall ijt . \beta(t, i, j) < t $$
		\item[\textit{S3)}] no link has a loss rate of \num{100}\%
			$$\forall ijt. \exists t' . \forall k. \beta(t' + k, i, j) \neq t $$
	\end{itemize}
\end{definition}

Nothing in S2 or S3 forbids the data flow function $\beta$ from delaying, losing
reordering or dublicating messages.

For a given schedule $(\alpha, \beta)$ and starting state $X$ we define $\delta$,
the asynchonous version of $\sigma$ as follows:

\begin{equation}
	\begin{align*}
		\delta^{0}(X)_{ij} &\triangleq X_{ij} \\
	\delta^{t}(X)_{ij} &=
	\begin{cases}
		\semiringchoice_{k}A_{ik}(\delta^{\beta(t,i,k)}(X)_{kj})\semiringchoice I_{ij}
		 & \text{if $i \in \alpha(t)$}, \\
		\delta^{t-1}(X)_{ij}
	     & \text{otherwise}, \\
 	\end{cases} 
	\end{align*}
\end{equation}

Giving what is demonstrated in \cite{daggitt2018asynchronous} if my implementation
is strictly incrasing then absolute convergence is guaranted.

\bibliographystyle{IEEEtran}
\bibliography{references}

\end{document}
