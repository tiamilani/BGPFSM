\documentclass[10pt,journal,onecolumn]{IEEEtran}

\usepackage[utf8]{inputenc}
\usepackage[T1]{fontenc}
\usepackage{silence}\WarningsOff[latexfont]

\usepackage{amsmath}
\usepackage{amsfonts}
\usepackage{amssymb}

\usepackage{graphicx}
\graphicspath{images/}
\usepackage{cite}
\usepackage{url}
\usepackage{subcaption}
\usepackage{float}
\usepackage[ruled,vlined,linesnumbered]{algorithm2e}
\SetKwProg{Fn}{Event}{}{}
\SetKw{And}{and}
\usepackage[binary-units,per-mode=symbol]{siunitx}
\sisetup{list-final-separator = {, and },detect-weight=true, detect-family=true}
\usepackage{booktabs}
\usepackage{pifont}
\usepackage{microtype}
\usepackage{textcomp}
\usepackage[american]{babel}
\usepackage[capitalise]{cleveref}
\def\figname{\csname cref@figure@name\endcsname\xspace}
\def\tabname{\csname cref@table@name\endcsname\xspace}
\def\secname{\csname cref@section@name\endcsname\xspace}
\def\eqpname{\csname cref@equation@name@plural\endcsname\xspace}
\crefname{algorithm}{Listing}{Lists.}
\Crefname{algorithm}{Listing}{Listings}
\SetAlgorithmName{Listing}{Listing}{List of Listings}
\crefname{lstlisting}{listing}{listings}
\Crefname{lstlisting}{Listing}{Listings}
\usepackage{xspace}
\usepackage{hyphenat}
\usepackage[draft,inline,nomargin,index]{fixme}
\fxsetup{theme=color}
\usepackage{grffile}
\usepackage{xfrac}
\usepackage{multirow}
%\usepackage[para]{footmisc}
\usepackage[font={small}]{caption}
\usepackage{imakeidx}

\usepackage{tikz}
\usetikzlibrary{calc,shapes,arrows,fit,positioning}

\usepackage{listings}
\lstset{
   language=sh,
   columns=fixed,
   breaklines=true,
   breakatwhitespace=true,
   prebreak=\textbackslash,
   basicstyle=\ttfamily\small,
   showstringspaces=false,
   upquote=true,
   keywordstyle=\ttfamily\small
}

\usepackage{color}
\definecolor{gray}{rgb}{0.4,0.4,0.4}
\definecolor{darkblue}{rgb}{0.0,0.0,0.6}
\definecolor{cyan}{rgb}{0.0,0.6,0.6}


\lstdefinelanguage{XML}
{
  morestring=[b]",
  morestring=[s]{>}{<},
  morecomment=[s]{<?}{?>},
  stringstyle=\color{black},
  identifierstyle=\color{darkblue},
  keywordstyle=\color{cyan},
  morekeywords={xmlns,version,type}% list your attributes here
}

\definecolor{codegreen}{rgb}{0,0.6,0}
\definecolor{codegray}{rgb}{0.5,0.5,0.5}
\definecolor{codepurple}{rgb}{0.58,0,0.82}
\definecolor{backcolour}{rgb}{0.95,0.95,0.92}

\lstdefinestyle{shell}{
    backgroundcolor=\color{backcolour},
    breakatwhitespace=false,
    breaklines=true,
    captionpos=b,
    keepspaces=true,
    showspaces=false,
    showstringspaces=false,
    showtabs=false,
    tabsize=2
}

% fix cleveref and breqn
\makeatletter
\let\cref@old@eq@setnumberOld\eq@setnumber
\def\eq@setnumber{%
\cref@old@eq@setnumberOld%
\cref@constructprefix{equation}{\cref@result}%
\protected@xdef\cref@currentlabel{%
[equation][\arabic{equation}][\cref@result]\p@equation\eq@number}}
\makeatother

% reduce verbatim font size
\usepackage{etoolbox}
\makeatletter
\patchcmd{\@verbatim}
  %%%{\verbatim@font} %% blow up TexMaker formatting ???!!! 
  {\verbatim@font\small}
  {}{}
\makeatother

\RequirePackage{xstring}
\RequirePackage{xparse}
\RequirePackage[index=true]{acro}
\NewDocumentCommand\acrodef{mO{#1}mG{}}{\DeclareAcronym{#1}{short={#2}, long={#3}, #4}}
\NewDocumentCommand\acused{m}{\acuse{#1}}


\acrodef{ADV}{advertisement}
\acrodef{AS}{Autonomous System}{short-plural=es}
\acrodef{BGP}{Border Gateway Protocol}
\acrodef{BIRD}{BGP Internet Routing Daemon}
\acrodef{DPC}{Destination Partial Centrality}
\acrodef{eBGP}{Exterior BGP}
\acrodef{ERP}{Exterior Routing Protocol}
\acrodef{IoF}{Internet on FIRE}
\acrodef{IP}{Internet Protocol}
\acrodef{MRAI}{Minimum Route Advertisement Interval}
\acrodef{NH}{Next Hop}
\acrodef{RFC}{Request For Comment} 
\acrodef{TCP}{Transmission Control Protocol}
\acrodef{FSM}{Finite State Machine}
\acrodef{DES}{Descrete Event Simulator}

\newcommand\useallac{
\acused{IP}
\acused{TCP}
\acused{RFC}
}

\useallac

\newcommand{\figwidthfour}{0.78}
\newcommand{\figwidth}{0.78}
\newcommand{\figvspace}{-1.5em}
\newcommand{\update}{\texttt{UPDATE}\xspace}
\newcommand{\nodeset}{\ensuremath{\mathcal{V}}\xspace}
\newcommand{\destinationset}{\ensuremath{\mathcal{C}}\xspace}
\newcommand{\edgeset}{\ensuremath{\mathcal{E}}\xspace}
\newcommand{\graph}{\ensuremath{\mathcal{G(\nodeset,\edgeset)}}\xspace}
\newcommand{\pathset}{\ensuremath{\mathcal{C}}\xspace}
\newcommand{\ascentg}{\ensuremath{\mathcal{G_{A}}\xspace}}
\newcommand{\ascentnodeset}{\ensuremath{\mathcal{V^{\ascentg}}}\xspace}
\newcommand{\ascentedgeset}{\ensuremath{\mathcal{E^{\ascentg}}}\xspace}
\newcommand{\ascentgraph}{\ensuremath{\mathcal{\ascentg(\ascentnodeset,\ascentedgeset)}}\xspace}
\newcommand{\dpc}{\ensuremath{\Delta}\xspace}
\newcommand{\tr}{\ensuremath{T_{R}}\xspace}

\newcommand{\tierg}{\ensuremath{\mathcal{G_{T}}\xspace}}
\newcommand{\tiernodeset}{\ensuremath{\mathcal{V^{\tierg}}}\xspace}
\newcommand{\tieredgeset}{\ensuremath{\mathcal{E^{\tierg}}}\xspace}
\newcommand{\tiergraph}{\ensuremath{\mathcal{\tierg(\tiernodeset,\tieredgeset)}}\xspace}
\newcommand{\descentg}{\ensuremath{\mathcal{G_{D}}\xspace}}
\newcommand{\descentnodeset}{\ensuremath{\mathcal{V^{\descentg}}}\xspace}
\newcommand{\descentedgeset}{\ensuremath{\mathcal{E^{\descentg}}}\xspace}
\newcommand{\descentgraph}{\ensuremath{\mathcal{\descentg(\descentnodeset,\descentedgeset)}}\xspace}

\IEEEoverridecommandlockouts
\makeindex[columns=3, title=Alphabetical Index, intoc]

\begin{document}

\title{User Manual BGP simulator}
\author{
	\IEEEauthorblockN{Mattia Milani}\\
    \IEEEauthorblockA{Dept. of Information Engineering and Computer Science, University of Trento, Italy}
    \texttt{mattia.milani@studenti.unitn.it}
}


\maketitle

\clearpage

\tableofcontents

\clearpage

\section{Guide environment}
\label{sec:guide_environment}

This guide has been tested on a clean installation of ubuntu 18.04
with python 3.6.9

\section{Experiments general chain}
\label{sec:general_chain}

This environment is composed by different components, thats executed one after
the other will produce the output expected.

\begin{figure}[h]
	\begin{center}
		\begin{tikzpicture}[scale=0.2]
			\tikzstyle{every node}+=[inner sep=0pt]
			\draw [black] (0,0) rectangle (10,5);
			\draw [black] (20,0) rectangle (30,5);
			\draw [black] (40,0) rectangle (50,5);
			\draw [thick, ->] (10,2.5) -- (20,2.5);
			\draw [thick, ->] (30,2.5) -- (40,2.5);
			\node [DES] at (5,2.5) {DES};
			\node [Analyzer] at (25,2.5) {Analyzer};
			\node [Plotter] at (45,2.5) {Plotter};
		\end{tikzpicture}
	\end{center}
	\caption{General environment chain}
	\label{fig:general_chain}
\end{figure}

\Cref{fig:general_chain} shows how in general an experiment is conducted.
The first step is the \ac{DES} that takes in input some configuration files
and runs each step of the simulation.
The network simulated evolves producting some output files in \textit{CSV} format
that would be the input for the next step.
The second part of this environment is take by the analyzer, it's the main
interpreter of the \ac{DES} output, it takes as input the CSVs files produced
by the first step. The output of the analyzer is another set of CSVs that are
easily plottable or that represent in a more clean way some aspectes of the simulations.
The analyzer can take as input multiple \ac{DES} execution and produce mean
results.
The analyzer is a necessary step to merge multiple experiments, it is also
necessary because from a single \ac{DES} execution is possible to study
a lot of different aspects (Time convergence, number of messages, a single node
evolution, etc.) and is not always necessary to study all this aspects.
The last part of an experiment is the plotter.
This part take as input CSV files created by the analyzer and simply plot what
is in them.

Each component is necessary to reach the goal of an experiment.

\section{How to install}
\label{sec:installation}

This section contains an explanation on how to install in the correct
way the software required by the environment.

In the main folder of the project is present a bash file that will install
everything you need, just use the following command:

\lstset{style=shell}
\begin{lstlisting}[language=bash]
	./install.sh
\end{lstlisting}

\section{Experiments configuration files}
\label{sec:exp_conf}

\subsection{Graph file}
\label{subsec:graph_file}

\subsection{Json file}
\label{subsec:json_file}

\section{Discrete event simulation}
\label{sec:des}

\subsection{Simulator parameters}
\label{subsec:des_param}

\subsection{First experiment}
\label{subsec:first_exp}

\subsection{Output of the experiment}
\label{subsec:des_output}

\section{Analyzer}
\label{sec:analyzer}

\subsection{Analyzer parameters}
\label{subsec:anal_param}

\subsection{Analyzer execution}
\label{subsec:anal_exec}

\subsection{Analyzer output}
\label{subsec:anal_output}

\section{Plot results}
\label{sec:plot_results}

\section{Graph generator}
\label{sec:graph_generator}

\subsection{Propery generators}
\label{sec:graph_propery_generators}

\section{Multiple experiments in parallel}
\label{sec:multiple_experiments}

\subsection{Parallel experiments examples}
\label{subsec:parallel_examples}

\section{Parallel experiments with multiple MRAIs}
\label{sec:parallel_mrais}

\subsection{Different MRAIs}
\label{subsec:MRAI_types}

\section{Examples}
\label{sec:examples}

\subsection{Line}
\label{subsec:ex_line}

\subsection{Line with delays}
\label{subsec:ex_line_delay}

\subsection{Line signaling}
\label{subsec:ex_line_signaling}

\subsection{Simple graph with MRAI}
\label{subsec:simple_graph_with_MRAI}

\subsection{Clique MRAI evolution}
\label{subsec:clique_evolution}

\subsection{Clique with RFD}
\label{subsec:clique_rfd}

\subsection{Clique different MRAIs comparison}
\label{subsec:clique_different_mrais}

\subsection{Small Internet graph}
\label{subsec:small_internet_graph}

\subsection{Small Internet graph with different MRAIs strategies}
\label{subsec:small_internet_graph_multiple_MRAIs}

\printindex
\bibliographystyle{IEEEtran}
\bibliography{references}

\end{document}
