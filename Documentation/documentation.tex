\documentclass[10pt,conference,letterpaper]{IEEEtran}

\usepackage[utf8]{inputenc}
\usepackage[T1]{fontenc}
\usepackage{silence}\WarningsOff[latexfont]

\usepackage{amsmath}
\usepackage{amsfonts}
\usepackage{amssymb}

\usepackage{graphicx}
\graphicspath{figures/}
\usepackage{cite}
\usepackage{url}
\usepackage{subfig}
\usepackage{float}
\usepackage[ruled,vlined,linesnumbered]{algorithm2e}
\SetKwProg{Fn}{Event}{}{}
\SetKw{And}{and}
\usepackage[binary-units,per-mode=symbol]{siunitx}
\sisetup{list-final-separator = {, and },detect-weight=true, detect-family=true}
\usepackage{booktabs}
\usepackage{pifont}
\usepackage{microtype}
\usepackage{textcomp}
\usepackage[american]{babel}
\usepackage[capitalise]{cleveref}
\def\figname{\csname cref@figure@name\endcsname\xspace}
\def\tabname{\csname cref@table@name\endcsname\xspace}
\def\secname{\csname cref@section@name\endcsname\xspace}
\def\eqpname{\csname cref@equation@name@plural\endcsname\xspace}
\crefname{algorithm}{Listing}{Lists.}
\Crefname{algorithm}{Listing}{Listings}
\SetAlgorithmName{Listing}{Listing}{List of Listings}
\crefname{lstlisting}{listing}{listings}
\Crefname{lstlisting}{Listing}{Listings}
\usepackage{xspace}
\usepackage{hyphenat}
\usepackage[draft,inline,nomargin,index]{fixme}
\fxsetup{theme=color}
\usepackage{grffile}
\usepackage{xfrac}
\usepackage{multirow}
%\usepackage[para]{footmisc}
\usepackage[font={small}]{caption}

\usepackage{tikz}
\usetikzlibrary{calc,shapes,arrows,fit,positioning}

\usepackage{listings}
\lstset{
   language=sh,
   columns=fixed,
   breaklines=true,
   breakatwhitespace=true,
   prebreak=\textbackslash,
   basicstyle=\ttfamily\small,
   showstringspaces=false,
   upquote=true,
   keywordstyle=\ttfamily\small
}

\usepackage{color}
\definecolor{gray}{rgb}{0.4,0.4,0.4}
\definecolor{darkblue}{rgb}{0.0,0.0,0.6}
\definecolor{cyan}{rgb}{0.0,0.6,0.6}


\lstdefinelanguage{XML}
{
  morestring=[b]",
  morestring=[s]{>}{<},
  morecomment=[s]{<?}{?>},
  stringstyle=\color{black},
  identifierstyle=\color{darkblue},
  keywordstyle=\color{cyan},
  morekeywords={xmlns,version,type}% list your attributes here
}


% fix cleveref and breqn
\makeatletter
\let\cref@old@eq@setnumberOld\eq@setnumber
\def\eq@setnumber{%
\cref@old@eq@setnumberOld%
\cref@constructprefix{equation}{\cref@result}%
\protected@xdef\cref@currentlabel{%
[equation][\arabic{equation}][\cref@result]\p@equation\eq@number}}
\makeatother

% reduce verbatim font size
\usepackage{etoolbox}
\makeatletter
\patchcmd{\@verbatim}
  %%%{\verbatim@font} %% blow up TexMaker formatting ???!!! 
  {\verbatim@font\small}
  {}{}
\makeatother

\RequirePackage{xstring}
\RequirePackage{xparse}
\RequirePackage[index=true]{acro}
\NewDocumentCommand\acrodef{mO{#1}mG{}}{\DeclareAcronym{#1}{short={#2}, long={#3}, #4}}
\NewDocumentCommand\acused{m}{\acuse{#1}}


\acrodef{ADV}{advertisement}
\acrodef{AS}{Autonomous System}{short-plural=es}
\acrodef{BGP}{Border Gateway Protocol}
\acrodef{BIRD}{BGP Internet Routing Daemon}
\acrodef{DPC}{Destination Partial Centrality}
\acrodef{eBGP}{Exterior BGP}
\acrodef{ERP}{Exterior Routing Protocol}
\acrodef{IoF}{Internet on FIRE}
\acrodef{IP}{Internet Protocol}
\acrodef{MRAI}{Minimum Route Advertisement Interval}
\acrodef{NH}{Next Hop}
\acrodef{RFC}{Request For Comment} 
\acrodef{TCP}{Transmission Control Protocol}

\newcommand\useallac{
\acused{IP}
\acused{TCP}
\acused{RFC}
}

\useallac

\newcommand{\figwidthfour}{0.78}
\newcommand{\figwidth}{0.78}
\newcommand{\figvspace}{-1.5em}
\newcommand{\update}{\texttt{UPDATE}\xspace}
\newcommand{\nodeset}{\ensuremath{\mathcal{V}}\xspace}
\newcommand{\destinationset}{\ensuremath{\mathcal{C}}\xspace}
\newcommand{\edgeset}{\ensuremath{\mathcal{E}}\xspace}
\newcommand{\graph}{\ensuremath{\mathcal{G(\nodeset,\edgeset)}}\xspace}
\newcommand{\pathset}{\ensuremath{\mathcal{C}}\xspace}
\newcommand{\ascentg}{\ensuremath{\mathcal{G_{A}}\xspace}}
\newcommand{\ascentnodeset}{\ensuremath{\mathcal{V^{\ascentg}}}\xspace}
\newcommand{\ascentedgeset}{\ensuremath{\mathcal{E^{\ascentg}}}\xspace}
\newcommand{\ascentgraph}{\ensuremath{\mathcal{\ascentg(\ascentnodeset,\ascentedgeset)}}\xspace}
\newcommand{\dpc}{\ensuremath{\Delta}\xspace}
\newcommand{\tr}{\ensuremath{T_{R}}\xspace}

\newcommand{\tierg}{\ensuremath{\mathcal{G_{T}}\xspace}}
\newcommand{\tiernodeset}{\ensuremath{\mathcal{V^{\tierg}}}\xspace}
\newcommand{\tieredgeset}{\ensuremath{\mathcal{E^{\tierg}}}\xspace}
\newcommand{\tiergraph}{\ensuremath{\mathcal{\tierg(\tiernodeset,\tieredgeset)}}\xspace}
\newcommand{\descentg}{\ensuremath{\mathcal{G_{D}}\xspace}}
\newcommand{\descentnodeset}{\ensuremath{\mathcal{V^{\descentg}}}\xspace}
\newcommand{\descentedgeset}{\ensuremath{\mathcal{E^{\descentg}}}\xspace}
\newcommand{\descentgraph}{\ensuremath{\mathcal{\descentg(\descentnodeset,\descentedgeset)}}\xspace}

\IEEEoverridecommandlockouts

\begin{document}

\title{BGPFSMSym WorkFlow plan}
\author{
	\IEEEauthorblockN{Mattia Milani\IEEEauthorrefmark{1}}
    \IEEEauthorblockA{\IEEEauthorrefmark{1}Dept. of Information Engineering and Computer Science, University of Trento, Italy}
    \texttt{mattia.milani@studenti.unitn.it}
}


\maketitle

\section{Main Idea}
\label{sec:mainIdea}
The main idea of this tool is to emulate and log the FSM of BGP nodes described
in \cite{griffinFSM}.
The tool should have a double function, emulate a random evolution of the nodes
grpah and the possibility to calculate all the possible states that a node can
assume from an input message.

\section{Goals}
\label{sec:goals}
This tool should help us to model the fact that \ac{MRAI} in \ac{BGP} is essential
to prevent the occurence of wedgies \fixme{insert citetion} or persistant 
instability situations.

\section{Inputs of the environment}
\label{sec:input}
The software will require two mandatory inputs:
\begin{itemize}
		\item Graph, this file would describe the graph and how is structured.
			a more deep explanation of the graph is provided in \cref{sec:graph};
		\item Environment descriptor, this file would describe the simulation
			environment, providing arguments to the simulator, this file is
			described more deeply in \cref{sec:envDesc}.
\end{itemize}

\section{Output of the environment}
The software output would depend on the type of the simulation:
\begin{itemize}
		\item Evolutional experiment this experiment would produce e simple CSV
			as output that can describe all the sates that a node assumes during
			the evolution.
		\item Compleate experiment, this type of experimetn goal is to describe
			all the possible states that nodes can assume, so the CSV presented
			as output will include all the states, even rare states that could
			happen.
\end{itemize}

\section{External Libraries and their use}
\label{sec:extLib}

\begin{itemize}
		\item ArgParse, used for the argument parsing;
		\item NetworkX, used for the graph/network handl;
		\item SimPy, could be used to simulate the entire environment.
		\item Pandas, data handling for easy manipulation of outputs in CSV
		\item ...
\end{itemize}

\section{Input Graph}
\label{sec:graph}
The input of the graph is fundamental for the software, its mandatory.
The graph is directed.
The graph file format is GraphML, a markup language for graphs.
Is possible to import and export graphml files from networks.
This format should give us the maximum possible degree of freedom.

Arguments of nodes (in red mandatory arguments):
\begin{itemize}
		\item \colorbox{red}{NodeId}: identifier of the node, would be treated
			as an integer, not integer Ids will raise an execption
		\item destination: in the future will be an ip prefix, for now it's a
			bool value that identify the node that will be reached from the 
			others.
\end{itemize}

Arguments for the edges (in red mandatory arguments):
\begin{itemize}
		\item delay: this argument will describe the delay that will be used
			for the communications on this link, is composed by two values
			min and max delay.
\end{itemize}

future arguments:
implementation of destinations like ip address, with the possibility to have
more ip addresses per node.
Insertion of \ac{MRAI} in the edge

\section{Input arguments}
\label{sec:envDesc}
It's mandatory to give an environment descriptor.
This file will be taken in consideration by the Argparse library to correctly
set up all the simulation variables and inputs.

All the arguments that could be used are:
\begin{itemize}
		\item graph, this argument describe the position of the graph file taken
			in consideration for the experiment
		\item experiment, this argument describe which type of experiment should
			be done, the possible values are "evolutional" "compleate"
		\item outputdir, this argument describe the directory that should be
			used as output for the CSV files, the if it does not exists will be
			created, if a directory with the same name already exists a new 
			directory will be created
		\item verbose, this argument describe how much verbose should be the
			output of the program
		\item delaydistribution, this argument describe the distribution that
			should be taken in consideration for the delays on the edges
		\item defaultdelay, this argumnet set the default delay min and max
			will be overrided by the graph argument if present
\end{itemize}

\section{Tests}
\label{sec:test}
Is possible to run the unit test that are present in the "test" folder to
check if the software is working properly

\section{Templates}
\label{sec:templates}
In the folder templates are present some templates experiments that could
be useful.

\section{Software main components}
\label{sec:softwarecomp}

The main components of the software are:
\begin{itemize}
		\item argument pareser, this component checks the arguments that are
			present in the environment file
		\item graph converter, this component is responsible to convert the
			graph passed as input in simpy components
		\item simulation environment, this is the main component for the simulation
			it takes as input the environment configuration and the components
			created for the simulation, is also responsible for the start and
			the stop of the simulation
		\item logger, this component is responsible for the inline output
			is possible to check the evolution of a simulation with the inline
			output
		\item CSV writer, this component is responsible for the output of the 
			system, the output must be a CSV in the predefined format.
\end{itemize}

Parallel execution of experiments thanks to \textit{parallel} \cite{Tange2011a}.

\bibliographystyle{IEEEtran}
\bibliography{references}

\end{document}
